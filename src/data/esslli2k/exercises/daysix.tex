\documentclass[10pt]{article}
\setlength{\evensidemargin}{0in}
\setlength{\oddsidemargin}{0in}
\setlength{\textwidth}{6.5in}
\setlength{\topmargin}{-0.5in}
\setlength{\textheight}{9in}
\pagestyle{empty}
\parindent=0pt

\usepackage{latexsym}
\usepackage{epsfig}
\usepackage{fancybox}
\usepackage{graphics}
\usepackage{graphpap}
\usepackage{tree-dvips}
\usepackage{avm}

\newcommand{\att}[1]{{\mbox{\scriptsize {\bf #1}}}}
\newcommand{\attval}[2]{{\mbox{\scriptsize #1}\ {{#2}}}}
\newcommand{\attvallist}[2]{{\mbox{\scriptsize #1}\ {<{#2}>}}}
\newcommand{\attvallistleft}[2]{{\mbox{\scriptsize #1}\ {<{#2}}}}
\newcommand{\attvallistright}[2]{{\mbox{\scriptsize #1}\ {{#2}>}}}
\newcommand{\attvaltlist}[2]{{\mbox{\scriptsize #1}\ {<{\myvaluebold{#2}}>}}}
\newcommand{\attvaltyp}[2]{{\mbox{\scriptsize #1}\ {\myvaluebold{#2}}}}
\newcommand{\attvalshrunktyp}[2]{{\mbox{\scriptsize #1}\ {\boxvaluebold{#2}}}}
\newcommand{\myvaluebold}[1]{{\mbox{\scriptsize {\bf #1}}}}
\newcommand{\boxvaluebold}[1]{{\fbox{\scriptsize {\bf #1}}}}
\newcommand{\ind}[1]{{\setlength{\fboxsep}{0.25mm} \: \fbox{{\tiny #1}} \:}}

\begin{document}
\begin{center}
\textbf{Advanced Grammar Engineering using HPSG: DAYS 1 \& 2}\\
ESSLLI 2000
\end{center}

\medskip
\textbf{Goals}
\begin{enumerate}
  \item Become familiar with the LKB grammar development environment.
  \item Perform list concatenation using difference lists.
  \item Add semantics to lexicon entries and rules.
  \item Use LKB generator to identify overgeneration in the grammar.
\end{enumerate}

\smallskip
\textbf{Preparation}
\begin{itemize}
  \item []
        There are two ways to obtain a starting grammar for this week;
        both require the following first step.  Participants of the
        introductory couse who have a {\em working\/} version of their
        grammar {\em including\/} lexical and inflectional rules can
        preserve their current types, lexicon, and rules files by
        applying the optional second and third steps below.
  \item [(i)]
        Retrieve a complete new grammar from the CVS version control
        system; in the {\em xterm\/} window type:\\
        \verb|  cvs checkout grammar6|\\
        This creates a new subdirectory `{\tt grammar6}' which contains
        a complete LKB grammar.
  \item []
        {\em Those of you who completed lexical and inflectional rules
        last week and want to preserve their grammars, perform steps
        (ii) and (iii) below; all others, go straight into the
        Exercises section.}
  \item [(ii)]
        Copy the files `{\tt types.tdl}', `{\tt lexicon.tdl}', 
        `{\tt rules.tdl}', `{\tt lrules.tdl}', and `{\tt inflr.tdl}'
        from your latest grammar into the `{\tt grammar6}' directory,
        e.g.:\\
        \verb|  cd grammar6|\\
        \verb|  cp ../grammar5/*.tdl .|\\
        Where you may need to substitute the correct directory name that
        contains your latest working grammar for `{\tt grammar5}'.
  \item [(iii)]
        Include the contents of the file `{\tt extras.tdl}' at the end
        of your own `{\tt types.tdl}' file.
        Move the attribute \att{ORTH} from {\it lex-item} to
        {\it syn-struc} and change its value from {\it string} to
        {\it *dlist*}.
        In the lexicon, replace all occurences of \att{ORTH} with the
        path to the first element of the new orthography difference
        list: \att{ORTH.LIST.FIRST} (a shorthand path notation).
        Make sure you can still load and run the resulting grammar
        before you move on.
\end{itemize}

\smallskip
\textbf{Exercises}
\begin{enumerate}
  \item The typed feature structure formalism that is implemented in
        the LKB (and similar systems) excludes relational constraints
        (like {\tt append()} or {\tt reverse()} of lists).  However,
        the {\em difference list\/} encoding in feature structures
        allows us to achieve list concatenation using pure
        unification.  A difference list is an open-ended list that is
        embedded into a container structure providing a `pointer' to
        the end of the list, e.g.\\
        \centerline{%
        \rlap{\oetag{A}}\oetavm{\mbox{*dlist*}$\,$}{
          \oeav{LIST}{\rlap{\oetag{1}}\oetavm{\mbox{*ne-list*}$\,$}{%
            \oeav{FIRST}{"foo"}\oeav{REST}{\oetag{2}*list*}}}
          \oeav{LAST}{\oetag{2}}}\hspace{1cm}
        \rlap{\oetag{B}}\oetavm{\mbox{*dlist*}$\,$}{
          \oeav{LIST}{\rlap{\oetag{3}}\oetavm{\mbox{*ne-list*}$\,$}{%
            \oeav{FIRST}{"bar"}\oeav{REST}{\oetag{4}*list*}}}
          \oeav{LAST}{\oetag{4}}}}\\
        Now, using the \att{LAST} pointer of difference list \oetag{A} we
        can append \oetag{A} and \oetag{B} by (i) unifying the front of
        \oetag{B} (i.e.\ the value of its \att{LIST} feature) into the
        tail of \oetag{A} (its \att{LAST} value) and (ii) using the tail
        of difference list \oetag{B} as the new tail for the result of
        the concatenation (see Section~4.8.2 in the on-line LKB manual).
  \item []
        The goal of this exercise is to pass up the \att{ORTH} values
        from words to phrases and use list concatenation to determine
        the value of \att{ORTH} at each phrase.  The top (`S') node
        in a complete analysis of a sentence should have as its
        \att{ORTH} value a difference list that contains all words of
        the sentence, preferably in the right order.
        \begin{itemize}
          \item [(a)]
                Make sure \att{ORTH} is an appropriate feature for the
                type {\it syn-struc} and is constrained to be of type
                {\it *dlist*} (see the preparatory instructions
                above); also, make sure that all lexical entries
                provide their orthography as the value of the feature
                path \att{ORTH.LIST.FIRST}.
          \item [(b)]
                The general constraint on \att{ORTH} (viz.\
                {\it *dlist*}) allows arbitrary-length lists.  For the
                list concatenation to work, it is essential that the
                \att{LAST} pointer in each difference list is correctly
                bound to the list position following the last element,
                i.e.\ for an empty list \att{LIST} and \att{LAST} need
                to be co-indexed, while for a singleton (one-element)
                list, \att{LAST} has to point to the same value as
                \att{LIST.REST}.
                Ensure that all lexical entries will have properly
                terminated singleton lists as their \att{ORTH} value by
                adding the appropriate constraint to {\it lex-item}
                (those who have started from our `grammar6' get this
                step for free). 
          \item [(c)]
                Each rule will be required to concatenate the
                \att{ORTH} values of all daughters to the rule and make
                the resulting list the \att{ORTH} value on the mother.
                To avoid duplication of the append operation in the
                `{\tt rules.tdl}' file, introduce types
                {\it unary-rule}, {\it binary-rule}, and
                {\it ternary-rule} that inherit from {\it phrase} and
                perform the concatenation of \att{ORTH} values.
          \item [(d)]
                In order to have one specific type for each actual rule
                in the `{\tt rules.tdl}' file, we need to
                cross-multiply the three types accounting for rule
                arity with the {\it head-initial} vs.\
                {\it head-final} distinction introduced last week.
                Use multiple inheritance to create types
                {\it unary-head-initial-rule},
                {\it binary-head-initial-rule}, and
                {\it ternary-head-initial-rule}, as well as
                {\it binary-head-final-rule}; for the time being, we
                have exactly one head-final rule in our
                grammar.
%          \item [(e)]
%                The type that our grammar uses as a `start symbol' is
%                called {\it root} and resides in `{\tt types.tdl}'. 
%                Based on this type, the LKB system validates each
%                complete analysis of an input string to unify with the
%                constraints on this type, thus checking for
%                `sentence-hood' (currently, only specifier-saturated
%                verbal projections are accepted as sentences).
%                Make sure that he {\it root} type can unify with an
%                appliation of the specifier$\,$--$\,$head rule which in
%                our grammar derives sentences (see (f) below).
          \item [(e)]
                Rework the file `{\tt rules.tdl}' to use the new, more
                specific rule types, as appropriate.
                Reload the grammar and check correctness by parsing a
                few sentences interactively and verifying that the
                \att{ORTH} value on `S' nodes contains all the words
                that contribute to the sentence.
                Run the `Batch parse' machinery on the `{\tt
                test.items}' file and validate the results.
        \end{itemize}
  \item Our approach to adding semantics to the grammar builds on list
        concatenation as the basic operation of composition.  Semantic
        relations are introduced by lexical entries and successively
        combined as words are combined with other words (or phrases) to
        form larger phrases.
        We will use a type {\it relation} to capture the basic
        units of semantics that are associated with words, and subtypes
        {\it arg1-relation}, {\it arg1-2-relation}, and
        {\it arg1-2-3-relation} for predicates of corresponding arity:\\
        \centerline{
        \oetavm{relation$\,$}{\oeav{PRED}{string}\oeav{ARG0}{index}}$\,$,
        \oetavm{arg1-relation$\,$}{\oeav{PRED}{string}
                                 \oeav{ARG0}{index}
                                 \oeav{ARG1}{index}}\hspace{2ex}
        $\cdots$%\hspace{2ex}
        \oetavm{arg1-2-3-relation$\,$}{\oeav{PRED}{string}
                                 \oeav{ARG0}{index}
                                 \oeav{ARG1}{index}
                                 \oeav{ARG2}{index}
                                 \oeav{ARG3}{index}}\hspace{2ex}}\\
        \begin{itemize}
          \item [(a)]
                Add atomic types {\it index} with subtypes
                {\it object} and {\it event}, which we will use to
                represent variables in role assignment.
          \item [(b)]
                Add the types {\it relation} through
                {\it arg1-2-3-relation} (as shown above) below 
                {\it feat-struc}.
        \end{itemize}
  \item []
        To associate semantics with words and phrases, we need another
        type that will serve as the value of a new \att{SEM} attribute
        in {\it syn-struc}:\\
        \verb|  |%
        \oetavm{semantics$\,$}{\oeav{INDEX}{index}
                            \oeav{KEY}{relation}
                            \oeav{RELS}{*dlist*}}\\
        Roughly speaking, the \att{INDEX} attribute corresponds to the
        external variable that is available for binding, the \att{KEY}
        attribute points to the distinguished relation that is used for
        semantic selection (typically contributed by the semantic head;
        see the MRS paper in the course reader), and the \att{RELS}
        attribute holds a list of relations (see below).
        For the lexical entries `dog' and `chase', respectively, we want
        the following semantics (as the value of their \att{SEM}
        feature):\\ 
         \oetavm{semantics$\,$}{%
           \oeav{INDEX}{\oetag{1}object}
           \oeav{KEY}{\rlap{\oetag{2}}\oetavm{\mbox{relation}$\,$}{%
             \oeav{PRED}{"dog\_rel"}
             \oeav{ARG0}{\oetag{1}}}}
            \oeav{RELS}{\oeavm{%
              \oeav{LIST}{\oeavm{\oeav{FIRST}{\oetag{2}}\oeav{REST}{\oetag{3}}}}
              \oeav{LAST}{\oetag{3}}}}}
         \oetavm{semantics$\,$}{%
           \oeav{INDEX}{\oetag{4}event}
           \oeav{KEY}{\rlap{\oetag{5}}1\oetavm{\mbox{arg1-2-relation}$\,$}{%
             \oeav{PRED}{"chase\_rel"}
             \oeav{ARG0}{\oetag{4}}
             \oeav{ARG1}{index}
             \oeav{ARG2}{index}}}
            \oeav{RELS}{\oeavm{%
              \oeav{LIST}{\oeavm{\oeav{FIRST}{\oetag{5}}\oeav{REST}{\oetag{6}}}}
              \oeav{LAST}{\oetag{6}}}}}\\
        \begin{itemize}
          \item [(a)]
                Introduce the type {\it semantics}, add the feature
                \att{SEM} to {\it syn-struc}, and constrain its value
                to {\it semantics}.
          \item [(b)]
                Enrich the {\it lexeme} type to reflect that (i)
                lexical items have a singleton \att{RELS} list, (ii)
                the \att{KEY} relation corresponds to the first (and
                only) element in \att{RELS}, and (iii) the \att{INDEX}
                is the \att{ARG0} of the \att{KEY}.
          \item [(c)]
                Enrich the types {\it det-lxm}, {\it noun-lxm}, and
                {\it verb-lxm} (or their equivalents) to constrain the
                semantic \att{INDEX} to be of type {\it object} (for
                determiners and nouns) and {\it event} (for verbs),
                respectively.
          \item [(d)]
                Add a unique relation, as the value of
                \att{SEM.KEY.PRED}, to each entry in the lexicon.
                Reload the grammar and use the `View -- Lex Entry' menu
                command to inspect the lexical entries for `dog' and
                `chase', making sure they look as specified above.
        \end{itemize}
  \item []
        Roughly similar to \att{ORTH}, each phrase will accumulate the
        semantic contributions from each of its daughters and use list
        concatenation in building up the \att{RELS} value; while
        technically we use a list for this purpose, the order of
        elements in \att{RELS} is actually irrelevant:\ we are using a
        list to represent a bag (or multi-set) of objects.
        Add the principles of semantic composition:
        \begin{itemize}
          \item [(a)]
                While unary rules simply pass up the entire \att{SEM}
                value from the daughter to the mother, binary and
                ternary rules apply difference list concatenation to
                the \att{SEM.RELS} values.
          \item [(b)]
                In each phrase, the \att{INDEX} and \att{KEY} values
                are contributed by the semantic head; for our grammar,
                the semantic head equals the syntactic head in all
                constructions: add appropriate co-indexation of
                \att{INDEX} and \att{KEY} to the {\it head-initial}
                and {\it head-final} types.
          \item [(c)]
                Inflectional rules are subtypes of {\it word} but
                behave much like unary rules; add the passing up of the
                \att{SEM} value from the daughter to that of the mother
                on the type {\it word}.
          \item [(d)]
                Reload the grammar, eliminate remaining errors (if
                any), and make sure that (i) coverage remains unchanged
                and (ii) the \att{RELS} value on `S' nodes contains the
                relations of all words in the input sentence.
        \end{itemize}
  \item []
        What remains to be achieved is the linking of syntactic
        arguments to semantic roles.  We will use the \att{INDEX} value
        of arguments (often called the {\em index\/}) to make this
        linking explicit. 
        This requires enriching the types for all lexemes that take
        arguments (i.e.\ have non-empty \att{SPR} or \att{COMPS} lists;
        likewise for non-empty \att{MOD} values on modifiers)
        to co-index the \att{INDEX} value of each argument with exactly
        one role in the relation of the functor (the semantic head).
        \begin{itemize}
          \item [(a)]
                Nouns identify the \att{INDEX} of their specifier with
                their own \att{INDEX} (also their \att{ARG0}).
          \item [(b)]
                All verbs identify the \att{INDEX} of their specifier
                with their \att{ARG1} role; in addition, transitive
                verbs identify the \att{INDEX} of their complement with
                their \att{ARG2}; analogously, ditransitive verbs map
                their complements to \att{ARG2} and \att{ARG3}.
          \item [(c)]
                Prepositions are special (and our current analysis is
                arguable anyway): they identify the \att{INDEX} of the
                {\it syn-struc} selected for by \att{MOD} with their
                own \att{INDEX}, and map the \att{INDEX} of their
                complement to \att{ARG1}.
          \item [(d)]
                Reload the grammar, make sure everything works, and
                admire the beauty of semantic composition.  Make sure
                to verify that in sentences like `the dog barks', `the
                cat gave that dog those aardvarks', or `the cat barked near
                those dogs' the following well-formedness conditions
                hold in the semantics: (i) all indexes are specialized to
                either {\it event} or {\it object}, (ii) determiner
                and noun in each noun phrase share the same \att{ARG0}
                variable, (iii) all roles in a verbal relation are
                bound to the indexes of the corresponding arguments,
                and (iv) the \att{ARG0} of a prepositional phrase 
                {\em modifier\/} is bound to the \att{ARG0} of the
                {\it event} or {\it object} being modified.
        \end{itemize}
  \item The LKB provides facilities to read out a semantic formula from
        a feature structure, print it in various formats, perform
        (limited) semantic inference, or generate from it.  Because our
        semantics is (currently) very shallow, however, only a subset
        of this functionality can be used meaningfully.
  \item []
        From the tree summary display (the window showing tiny trees),
        execute the commands `MRS' and `Indexed MRS' to see a readable
        form of the semantics produced by your grammar.
  \item 
        The LKB comprises a chart-based generator that, given a
        suitable semantic formula, can generate all sentences (i.e.\
        strings) that correspond to this semantics.  
        For the generator to work, we need to make a final extension to
        our grammar:
        \begin{itemize}
          \item [(a)]
                For generator-internal purposes, the generator requires
                \att{INDEX} values to have one appropriate feature,
                \att{INSTLOC}, whose value is of type {\it instloc}.
                Add the atomic type {\it instloc} (as a subtype of
                {\it *top*}) and enrich the type definition for
                {\it index}.  Reload the grammar.
          \item [(b)]
                To get full access to the LKB generation component,
                execute the `Options -- Expand Menu' command.  Then, to
                make the lexicon accessible to the generator, execute
                `Generator -- Index'; if there are warnings or errors
                at this stage, revisit your grammar.
                In order for the indexing to happen at load time, edit
                the file `{\tt script}' in your grammar directory which
                is the load file for the grammar.  Towards the
                end of the file, uncomment the line 
                {\tt (index-for-generator)} by removing all semicolons.
        \end{itemize}
        \item []
        Once indexing is complete, from the tree summary display
        explore the output of the `Generate' 
        command for various sentences; this is a short-cut command for
        extracting the indexed semantics from a feature structure and
        using that as the input for the generator.
        At this point, generating from the semantics of `the aardvark
        barks' should return four sentences:\ investigate the reasons
        for this overgeneration, identify the missing bits of
        information, and speculate about resolving this issue.
\end{enumerate}
\end{document}



