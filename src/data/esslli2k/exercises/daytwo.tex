\documentclass[10pt]{article}
\setlength{\evensidemargin}{0in}
\setlength{\oddsidemargin}{0in}
\setlength{\textwidth}{6.5in}
\setlength{\topmargin}{-0.5in}
\setlength{\textheight}{9in}
\pagestyle{empty}
\parindent=0pt

\begin{document}
\begin{center}
\textbf{An Introduction to Grammar Engineering using HPSG: DAY 2}\\
ESSLLI 2000
\end{center}

\medskip
\textbf{Goals:}
\begin{enumerate}
\item Learn about typed feature structures and unification.
\item Expand the grammar to capture agreement constraints.
\item Expand the grammar to provide an analysis of modification.
\end{enumerate}

\smallskip
\textbf{Exercises:}
\begin{enumerate}
\item Bring up
the LKB and load grammar2:
\begin{enumerate}
\item Check out a copy of grammar2 by typing {\tt cvs checkout grammar2} in 
an xterm window. 
\item Start emacs and in the Emacs window:
\par\texttt{<Esc> x lkb}
\item Load the grammar by selecting \textsf{Load / Complete
grammar} in the 
``Lkb Top'' window, then double-clicking on the directory ``grammar2'' and on the
file ``script''. 
\end{enumerate}
\item Test the grammar by parsing the sentence {\it The cat gave that dog to 
those dogs.}
\item Extend the grammar to capture subject-verb agreement, admitting e.g.
{\it The dog barks.} but not {\it *The dogs barks.}.  We will introduce 
constraints on the {\sc spr} attribute of lexical entries requiring that the
person-number properties match between head and specifier.  Rather than using
separate features for number and person, we will use types that combine both
properties, allowing a more direct encoding of English inflectional morphology.
\begin{enumerate}
\item Add this small type hierarchy to the file types.tdl, making {\it pernum}
a subtype of {\it feat-struc}:
\begin{verbatim}
                          			pernum
                           /    \
                       3sing   non-3sing
\end{verbatim}
\item Also in the file types.tdl, add the feature {\sc agr} to the type 
{\it pos}, with its value constrained to be of the type {\it pernum} that you 
just added. 
\item In the file lexicon.tdl, add the appropriate constraint to each verb,
restricting the {\sc agr} value of its {\sc spr}.
\item Also in the file lexicon.tdl, add the correct {\sc agr} value to each
noun.
\item Save your changes, then reload the grammar, apply the batch test with
the file ``agr.items'', examine your results, and make any necessary 
corrections.
\end{enumerate}

\item Extend your analysis to cope with determiner-noun agreement, admitting
e.g. {\it These dogs bark.} but not {\it These dog barks.}.

\begin{enumerate}
\item In the file lexicon.tdl, modify each noun's lexical entry by adding
the appropriate constraint on the {\sc agr} value of its {\sc spr}.
\item Also in lexicon.tdl, add the correct {\it agr} value to each determiner.
\item Check your revised grammar again using the file ``agr.items'', and make
any necessary corrections.  
\item Add some additional test examples to this file with varied combinations
of mismatch in agreement among determiners, nouns, and verbs.  Then run the
batch test and examine the results.
\end{enumerate}

\vspace{.5in}

\item Extend the grammar to provide an analysis of modification, admitting
sentences like {\it The dog barks near the cat.}  We introduce a new head 
feature {\sc mod} and a new syntactic rule for modifiers, and we make use 
of the notion of underspecification.
\begin{enumerate}
\item In the file types.tdl, add the feature {\sc mod} to the definition of
the type {\it pos}, with its value constrained to be of *list*, the same
type as for the attribute {\sc spr}.
\item Also in types.tdl, assign an appropriate value for {\sc mod} to each
of the subtypes of {\it pos}.
\item In the file rules.tdl, add a new rule ``head-modifier'' rule somewhat
similar to the head-specifier rule, but with the modifier daughter constraining
the head daughter.
\item In the file lexicon.tdl, add a lexical entry for the preposition 
{\it near}.
\item Save your changes, then test your revised grammar using the file ``mod.items''.  Examine the results, and make any necessary corrections.
\item If your analysis does not already admit examples like {\it The dog near 
the cat barks.}, modify your grammar appropriately.
\item If your analysis provides two parses for the sentence {\it The dog barks near the cat}, modify your grammar to eliminate one of the two parses, then run
the batch parse again with the file ``mod.items'', and examine the results.
\item Add additional sentences to the file ``mod.items'', and notice what 
happens to the number of analyses as you add prepositional phrase modifiers 
within a single sentence.
 \end{enumerate}

 \end{enumerate}
\end{document}
