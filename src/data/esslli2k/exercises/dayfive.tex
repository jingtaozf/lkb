\documentclass[10pt]{article}
\setlength{\evensidemargin}{0in}
\setlength{\oddsidemargin}{0in}
\setlength{\textwidth}{6.5in}
\setlength{\topmargin}{-0.5in}
\setlength{\textheight}{9in}
\pagestyle{empty}
\parindent=0pt

\begin{document}
\begin{center}
\textbf{An Introduction to Grammar Engineering using HPSG: DAY 5}\\
ESSLLI 2000
\end{center}

\medskip
\textbf{Goals:}
\begin{enumerate}
\item Add a lexical rule
\item Rework the grammar to make the lexeme / word distinction and
add inflection.
\end{enumerate}

\smallskip
\textbf{Exercise A:}
Add the dative lexical rule, according to the instructions in yesterday's 
handout.\\
\textbf{Exercise B:}
Rework the grammar in order to make the distinction between 
lexemes and words and to
add inflection in order 
to remove redundancy in the lexicon.  Following the Sag and Wasow textbook, we will
make every entry in the lexicon a subtype of lexeme, but require that the
grammar rules operate on structures which are subtypes of word (as with the current grammar).  This means that every lexical entry will have to be converted from
lexeme to word via a rule.  Some of these rules (plural etc) have morphological
effect.

This exercise involves a considerable amount of rearrangement 
of the grammar files.  This is, unfortunately, something
that grammar engineers end up doing quite often in real life.
Take things step-by-step, and make backup copies of your files.
You can do this by executing a {\tt cp -r} command in the xterm
window --- e.g.,
\begin{verbatim}
cp -r <mygrammar> <mygrammar>-bk1
\end{verbatim}

\begin{enumerate}
\item Add the top level lexeme / word distinction.   
We do this by distinguishing between two types, {\bf word} and {\bf lexeme}
which inherit from a new type {\bf lex-item}.  The feature ORTH
is introduced on {\bf lex-item}.  The type {\bf word}
will be a supertype of all the lexeme-to-word rules.  The structures you
need are given below.
\begin{verbatim}
lex-item := syn-struc &
[ ORTH string ].

lexeme := lex-item.

word := lex-item &
[ HEAD #1,
  SPR #2,
  COMPS #3,
  ARGS < lexeme &
         [ HEAD #1,
           SPR #2,
           COMPS #3 ]
         > ].
\end{verbatim}
\item The old types {\bf verb-word} etc had constraints which 
should belong to lexemes under our new view of the world.
Add subtypes of {\bf lexeme} for {\bf verb-lxm}, 
{\bf noun-lxm} etc, which use the constraints
from the old {\bf verb-word}, {\bf noun-word} etc.
Make {\bf det-lxm} and {\bf prep-lxm}
both inherit from a type {\bf const-lxm}, defined as follows:
\begin{verbatim}
const-lxm := lexeme.
\end{verbatim}
\item Add types {\bf noun-form} and {\bf verb-form} which inherit from type
{\bf word}.  These are effectively supertypes
of inflectional rules: their ARGS value should be constrained to
be of the corresponding lexeme type. 
\item  The old type {\bf verb-word}
was a supertype of two sorts of information which are 
inconsistent on our new view.
Modify your type file in order to
split the types that were a subtype of {\bf verb-word}
between those which are appropriate for lexemes
and those which are to do with inflections.  For instance, your 
type for transitive verbs will be a subtype of {\bf verb-lxm}
while your type for past verbs will be a subtype of {\bf verb-form}.
You should remove all types which inherit from both
the inflectional types and the lexeme types: 
their place will be taken by lexical
rules.
Similarly, redistribute all the types which inherit from 
{\bf noun-word} etc.
Notice that the types which inherit from {\bf word} are now effectively
defining different inflectional rules.
\item Modify your lexicon.tdl file:
\begin{enumerate}
\item Remove all entries which
are not morphologically equivalent to
base forms (e.g., {\it dogs} etc).  If your file has {\it gave}
but not {\it give}, replace the entry for {\it gave} with one for {\it give}.
\item Change the types on the remaining entries so they are all
subtypes of lexeme appropriate for the entry.  Your file should
now consist of base forms with just base orthography and a single type.
\end{enumerate}
\item The dative shift rule is now a lexeme-to-lexeme rule, so you will need
to edit lrules.tdl appropriately.
Your rules.tdl file should stay unchanged.
\item At this point you should have a grammar which loads but does not parse anything, because we have not yet added any inflectional rules.  Check
that the grammar loads correctly before proceeding to the next step.
\item 
\begin{verbatim}
cvs update
\end{verbatim}
We have given you an
inflr.tdl file which defines the actual inflectional rules.
Please do not worry about the lines beginning with \%, these are just instructions to the orthographemic component.  The only change you will need to make to
inflr.tdl is to replace the types {\bf past-verb}, {\bf plur-verb},
{\bf sing-verb}, {\bf plur-noun} and {\bf sing-noun} with the types that
you have already defined (i.e., those types which are now subtypes of
{\bf verb-form} and {\bf noun-form}).
\item Check to ensure that your grammar's coverage remains the same.
\end{enumerate}
\end{document}
