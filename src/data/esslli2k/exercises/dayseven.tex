\documentclass[11pt]{article}
\usepackage{gb4e}
\setlength{\evensidemargin}{0in}
\setlength{\oddsidemargin}{0in}
\setlength{\textwidth}{6.5in}
\setlength{\topmargin}{-0.5in}
\setlength{\textheight}{9in}
%\pagestyle{empty}
%\parindent=0pt

\begin{document}
\begin{center}
\textbf{Advanced Grammar Engineering using HPSG: DAY 2$\frac12$}\\
ESSLLI 2000
\end{center}

\smallskip
\noindent\textbf{Exercises}

Esperanto is an artificial language created by the Polish
oculist L.L.~Zamenhof in the nineteenth century. Zamenhof designed the
language to have a very simple and regular morphology, on the theory that war
and pestilence were caused by irregular verbs. While his dream of world peace
has not been achieved, his language has turned out to be very useful for
constructing problem sets in introductory linguistics classes.

For tomorrow's in-class exercise, you will be applying the grammar engineering
techniques you have mastered for English to a small grammar of Esperanto.  To
get ready for that, try out this pencil-and-paper exercise tonight. Based on the
sentences in (1), write a small English-Esperanto morpheme dictionary, and use
it to translate the English sentences in (2) into Esperanto.  You won't need to
turn in your results, but working through this will give you a feel for the
language and you will find the Esperanto grammar easier to follow tomorrow.
\begin{exe}
\ex
\begin{xlist}
\ex
La alta knabo malsani\^gis.\\
`The tall boy fell ill.'
\ex 
\^Cu li grandigis la grandecon de la dormejo?\\
`Did he increase the size of the dormitory?'
\ex Anka\v{u} malaltaj knabinoj povas esti belaj.\\
`Short girls, too, can be beautiful.'
\ex Mia patro estas sana \^car li ne trinkas vinon.\\
`My father is healthy because he doesnt drink wine.'
\ex
La bonaj mona\^hinoj volis pre\^gi en la pre\^gejo.\\
`The good nuns wanted to pray in church.'
\ex Lerni la esparantan lingvon estas facila.\\
`It's easy to learn Esperanto.'
\ex Mi vidis \^sian onklon en la trinkejo.\\
`I saw her uncle in the bar.'
\ex La beleco de la lingvo estas \^gia facileco.\\
`The beauty of the language is its simplicity.'
\ex \^Cu vi konas miajn onklojn?\\
`Do you know my uncles?'
\end{xlist}
\ex
\begin{xlist}
\ex Did her aunt know my mother?
%\^Cu \^sia onklino konis mian patrinon?
\ex His health has deteriorated.
%Lia saneco malgrandi\^gis.
\ex The boys can also learn difficult languages at school.
%Anka\v{u} knaboj povas lerni malfacilajn lingvojn en la lernejo. 
\ex The monks adorned the church.
%La mona\^hoj beligis la pre\^gejon.
\ex Does your mother want to put the boys to sleep?
%\^Cu via patrino volas dormigi la knabojn
\end{xlist}
\end{exe}
The first step in analyzing a new language like this is to figure out which
words in Esperanto match up to which words in the English translations. Look
for sentences with the same or related words in the English translation, and
then look for words which look similar in the corresponding Esperanto
sentences.  In this case, it's pretty easy, since Esperanto's syntax is almost
exactly the same as that of English (Zamenhof didn't have a lot of
imagination).  To take the first sentence as an example, we can break it down
like this:
\begin{center}
\begin{tabular}{|l|l|l|l|}
\hline
La & alta & knabo & malsani\^gis.\\
\hline
The & tall & boy & fell ill\\
\hline
\end{tabular}
\end{center}
Once you have a good idea what the rules for constructing sentences are, look
for patterns across individual words.  Find words with similar meanings or
similar grammatical functions, and see if there are common elements in their
spellings. Keep in mind that Esperanto has no irregular morphology, so stems
and endings always have exactly one spelling.  Working from the first example
again, you might notice that adjectives often end in \emph{a} and nouns often
end in \emph{o}, so these would be good candidates for adjective and noun
endings. You might also notice a similarity between \emph{malsani\^gis} `fell
ill' and \emph{malaltaj} `short' in example (1c), \emph{sana} `healthy' in
(1d), and \emph{volis} `wanted' in example (1e).  From this you might guess
that the structure of this word is:
\begin{center}
\begin{tabular}{rcl}
\begin{tabular}{|l|l|l|l|}
\hline
mal & san & i\^g & is\\
\hline
not & healthy & become & past\\
\hline
\end{tabular}  & = & became not healthy
\end{tabular}
\end{center}
Continue in the same way with all the words, and you will collect an inventory
of morphemes that you can use to construct the new words you need to translate
the sentences in (2).
 
\end{document}
