\documentclass[11pt]{article}
\usepackage{gb4e}
\usepackage{avm}
\setlength{\evensidemargin}{0in}
\setlength{\oddsidemargin}{0in}
\setlength{\textwidth}{6.5in}
\setlength{\topmargin}{-0.5in}
\setlength{\textheight}{9in}
%\pagestyle{empty}
\parindent=0pt

\begin{document}
\begin{center}
\textbf{Advanced Grammar Engineering using HPSG: DAY 3}\\
ESSLLI 2000
\end{center}

\smallskip
\noindent\textbf{Background}\\
Here are some possible translations of the English sentences from Day
2$\frac12$:
\begin{exe}
\ex
\begin{xlist}
\ex \^Cu \^sia onklino konis mian patrinon?\\
`Did her aunt know my mother?'
\ex Lia saneco malboni\^gis.\\
`His health has deteriorated.'
\ex La knaboj anka\v{u} povas lerni malfacilajn lingvojn en la lernejo.\\
`The boys can also learn difficult languages at school.'
\ex La mona\^hoj beligis la pre\^gejon.\\
The monks adorned the church.
\ex \^Cu via patrino volas dormigi la knabojn\\
`Does your mother want to put the boys to sleep?'
\end{xlist}
\end{exe}
If these don't make sense to you, let one of us know. 

Before you start today's exercise, check out a small partial grammar of
Esperanto:
\begin{verbatim}
cvs checkout esperanto
\end{verbatim}
This grammar should look very familiar---it is very much like the English
grammar you started out with on Monday. The lexicon has been replaced with an
inventory of Esperanto roots, things have been set up for the more complex
nominal morphology, and a couple of rules have been added. Also, to make
typing easier, letters which have accents in standard Esperanto orthography
are written as double letters for the purposes of this grammar.  So, that
means \emph{\^cu} is written \texttt{ccu} and \emph{anka\v{u}} is written
\texttt{ankauu}. Take a minute to get familiar with the grammar and figure out 
how it all works.

One thing to pay special attention to is different type of lexical rules.
For English, we had a distinction between derivational rules which mapped
lexemes to lexemes, and inflectional rules which mapped lexemes to words.
This general plan will work for Esperanto verbs as well, since while they may
have several derivational affixes (added by lexeme to lexeme rules), they only
have one inflectional affix (added by a lexeme to word rule).  For Esperanto
nouns and adjectives, though, the situation is more complicated.  A noun like
\emph{knabojn} `boys (\textsc{acc})' consists of a root \emph{knab} plus three 
inflectional endings \emph{o}, \emph{j}, and \emph{n}.  To handle this we need 
to add new kinds of lexical rules that we'll call lexeme to stem, stem to
theme, and theme to word lexical rules.  Every noun and adjective lexeme has
to go through each of these layers before it can become a word:
\begin{center}
\begin{tabular}{lcl}
knab (lexeme) + o & = & knabo (stem)\\
knabo (stem) + j & = & knaboj (theme)\\
knaboj (theme) + n & = & knabojn (word)
\end{tabular}
\end{center}
In addition, nouns and adjectives, like verbs, can undergo derivational
(lexeme to lexeme) rules, adding even more complexity:
\begin{center}
\begin{tabular}{lcl}
knab (lexeme) + in & = & knabin (lexeme)\\
knabin (lexeme) + o & = & knabino (stem)\\
knabino (stem) + j & = & knabinoj (theme)\\
knabinoj (theme) + n & = & knabinojn (word)
\end{tabular}
\end{center}
As in the English grammar, we control which rule can apply to which form
using a set of lexical types. Play around with this a little to see how it
works, but keep in mind that not all the lexical rules are in place yet.  Some 
of them you'll be adding in the following exercises.

\smallskip
\noindent\textbf{Exercises}
\begin{enumerate}
\item One linguistic device that plays an important role in Esperanto but
almost none in English is case marking.  Nouns which function as a direct
object take the accusative suffix \emph{-n}, and adjectives and possessive
determiners agree in both case and number with the noun they modify.
\begin{enumerate}
\item Add a feature \textsc{case} to \emph{nominal} which takes a feature
structure of type \emph{case} as its value.  Add \emph{nom} and \emph{acc} as 
subtypes of \emph{case}.  
\item Add constraints to the relevant lexeme types to enforce case assignment.
Verbs select for a nominative specifier and (if transitive) an accusative
complement, and prepositions select for a nominative complement.
\item Add case agreement constraints so that determiners and adjectives agree
with the noun in case as well as number.
\item Add the lexical rules for case marking.  Since the case marker comes
last in the word, case marking rules are theme to word lexical rules.  The
grammar you checked out has a lexical rule \emph{no-case-word-rule} in
\texttt{inflr.tdl} and a corresponding type \emph{no-case-word} in
\texttt{types.tdl}.  This rule maps any theme to a word without adding a case
ending.  Delete this rule (it's just a placeholder so you can load the
grammar), and add the correct lexical rules which map a theme to either a
nominative word or an accusative word, with the appropriate ending.  You will
probably need two rules, one for each case.
\item Check your work by parsing the sentences in \texttt{case.items}.
\end{enumerate}
\item Next you need to add the possessive determiners.  Esperanto possessives
are tricky because syntactically they behave like determiners, but
morphologically they behave like adjectives.
\begin{enumerate}
\item Create a subtype of \emph{det} called \emph{poss}.  Since the
inflectional rules for adjectives will have to apply to either adjectives or
possessives, add \emph{adj-or-poss} as a subtype of \emph{nominal} and make 
both \emph{adj} and \emph{poss} inherit from it.
\item Modify \emph{adj-stem} so that the \emph{adj-stem-rule} will apply to 
both adjectives and possessives.
\item Add a lexeme type for possessive determiners.
\item Add new lexical entries for the possessive forms of the pronouns in
\texttt{lexicon.tdl}. (Or, if you are feeling ambitious, set up a lexical rule 
to derive possessives from pronouns, or to derive pronouns from possessives.)
\item Check that your modifications have had the desired effect by parsing the 
examples in \texttt{poss.items}.
\end{enumerate}
\item Esperanto also has a number of lexeme to lexeme lexical rules.  Some,
like \emph{mal-} prefix rule, map a lexeme to a lexeme of the same category,
but with a different meaning.  Others, like the \emph{-igg} suffix rule, map a 
lexeme of one category to a lexeme of a different category.  Following the
example of the \emph{neg-adj-rule} and the \emph{inchoative-verb-rule}, add
lexical rules for the suffixes \emph{-in}, \emph{-ej}, \emph{-ec}, and
\emph{-ig}. You may find it easer to add two separate rules for \emph{-ig},
one mapping adjectives to verbs and one mapping verbs to verbs.  Test your
results with \texttt{endings.items}.
\item Infinitives (like \emph{lerni} `to learn') are verb forms, but
infinitive phrases (like \emph{lerni la esparantan lingvon} `to learn the
Esperanto language') are a lot like noun phrases in that they can be the
subject or object of a verb.  Add a lexeme to word lexical rule for
infinitives which forms a noun word from a verb lexeme.  The \textsc{head}
value of the output should be \emph{noun}, but the \textsc{comps} value should 
be the same as the underlying verb and the \emph{spr} value should be empty.
Test your results with \texttt{inf.items}.
\item Add new lexeme types and lexical entries for the particles \emph{\^cu},
\emph{ne}, and \emph{anka\v{u}}.  These particles should combine with the
phrases they attach to via the head-complement rule.  Make a set of test items
to verify your analysis.
\item Finally, try parsing the sentences from last night's exercise in
\texttt{hard.items}.  Does everything work?  If not, try to figure out what's
going wrong.  If so, then great: you've now got a fairly wide coverage
syntactic grammar of Esperanto.  All that's left to do is add semantics.  (If
you have time today, see how far you can get with it.  Otherwise, that's maybe
a project to come back to on some snowy February afternoon.)
\end{enumerate}
\end{document}

