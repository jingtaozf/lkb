\documentclass[10pt]{article}
\setlength{\evensidemargin}{0in}
\setlength{\oddsidemargin}{0in}
\setlength{\textwidth}{6.5in}
\setlength{\topmargin}{-0.5in}
\setlength{\textheight}{9in}
\pagestyle{empty}
\parindent=0pt

\begin{document}
\begin{center}
\textbf{An Introduction to Grammar Engineering using HPSG: DAY 1}\\
ESSLLI 2000
\end{center}

\medskip
\textbf{Goals:}
\begin{enumerate}
\item Become familiar with the LKB grammar development system.
\item Learn to extend a grammar by adding lexical entries.
\item Expand the grammar to deal with a ditransitive verb.
\item Expand the grammar to add prepositional phrase arguments.
% head modifier rule - if anyone looks like they're getting bored!
\end{enumerate}

\smallskip
\textbf{Exercises:}
\begin{enumerate}
\item Bring up
the LKB and load grammar1:
\begin{enumerate}
\item Checkout a copy of grammar1 by {\tt cvs checkout grammar1} in 
an xterm window. 
\item Start emacs and in the Emacs window:
\par\texttt{<Esc> x lkb}
\item Load the grammar by selecting \textsf{Load / Complete
grammar} in the 
``Lkb Top'' window, then double-clicking on the directory ``grammar1'' and on the
file ``script''.  Reassuring messages will appear in the ``Lkb Top'' window,
and a window will pop up showing you the type hierarchy for this small
grammar.
\end{enumerate}
\item Test the system by parsing the sentence {\it the cat chased the dog}:
\begin{enumerate}
\item With the mouse in the ``Lkb Top'' window, click on the button
\textsf{Parse}. 
\item Click on the menu item \textsf{Parse input\ldots}.
\item Type in the sentence {\it the cat chased the dog} replacing
the existing contents of the window.
\item Click on the
button \textsf{OK}.
\end{enumerate}
The system will parse the sentence and pop up a window containing a little 
parse tree for the single analysis of this sentence.  Click on the parse
tree to get a menu which allows you to enlarge it and look at the nodes.
\item Try the simple batch parsing mechanism:
\begin{enumerate}
\item In the ``Lkb Top'' window, click on the button \textsf{Parse} and
then on the menu item \textsf{Batch parse\ldots}, which will pop up a window
asking you for the file to be processed.
\item Click on the file ``test.items'' in your grammar directory. then hit
the button \textsf{OK}.  This will pop up a new window asking you for the name
of the file where the results of the batch run will be stored.
\item Enter the name ``test.results'' and hit the \textsf{OK} button.
The system will print the message \texttt{Parsing test file} in the 
{\tt *common-lisp*} emacs buffer when it starts, and will print the message \texttt{Finished test
file} when it is done.
\item Open the file ``test.results'' from emacs
and inspect the parsing results.
\end{enumerate}
\item Add a lexical entry for another animal noun of your choice:
\begin{enumerate}
\item In the Emacs window, open the file ``lexicon.tdl'' for editing
(see the emacs handout)
\item In the lexicon.tdl buffer that you get, copy the five lines that
define the lexical entry for \textbf{dog} and modify your copy to make the
value of \textsc{orth} appropriate for another animal.
\item Save the changed version of the file.
\end{enumerate}
\item Reload the grammar and test the effect of your addition:
\begin{enumerate}
\item In the ``Lkb Top'' window, select \textsf{Load / Reload
grammar}.
\item Parse the sentence {\it the cat chased the
$\langle\mbox{your-animal}\rangle$}
\item Add this sentence to the test.items file and rerun the batch
check.
\end{enumerate}
\item Investigate the grammar in order to get an intuitive idea 
of how it works (formal details
will be discussed later).  In particular, look at the following sentences and 
try and decide why they do or do not parse:
\begin{itemize}
\item the cat barks
\item the cat chased
\item cat barks
\item the cat bark
\item bark
\end{itemize}
Note that the parse chart will be available even if the sentence didn't parse,
and you can click on the nodes in that to display feature structures.
Notice that the grammar is parsing some sentences incorrectly
(overgeneration)
and failing to parse some sentences that should parse (undergeneration).
\item The rule that is needed for ditransitives is in the grammar,
but there are no lexical entries that utilize it.  Add an entry for {\it gave} which takes two noun phrase
complements (i.e., the entry you would need to parse {\it that dog gave the cat the $\langle\mbox{your-animal}\rangle$}). 
\begin{enumerate}
\item Copy the entry for {\it chased} in lexicon.tdl
\item Replace the orthography value as before.
\item Add an extra element to the COMPS list, which will be a duplicate of the
one that is already there.  Note: lists are delimited by
{\tt <} and {\tt >} and that elements on lists are
separated by commas.
\item Test by parsing {\it that dog gave the cat the $\langle\mbox{your-animal}\rangle$}.  Also test for overgeneration by confirming that you cannot
parse {\it that dog gave the cat}.  
\item Add appropriate sentences to test.items
\end{enumerate}
\item Add a new type and two new lexical entries 
in order to parse
{\it that dog gave the cat to the $\langle\mbox{your-animal}\rangle$}
\begin{enumerate}
\item Add the type {\bf prep} as
a new subtype of the type {\bf pos} to the file types.tdl, by copying 
the type description for {\bf noun} and replacing {\bf noun} with {\bf prep}. 
\item Add a lexical entry for the preposition {\it to}.  This should be similar
to the entry for {\it chased} in that {\it to} will take a single noun
phrase complement, but the value for HEAD should be {\bf prep} and the
value of SPR should be the empty list (i.e., {\tt <>}).
\item Add another lexical entry for {\it gave}.  You can copy your existing entry for {\it gave} but
you will need to use
a different identifier (i.e., the thing to the left of the \verb!:=!), e.g., 
\verb!gave_2!.  You also need to change the second element in
COMPS to make this entry require a PP (we won't bother about
making sure it's a {\it to}-PP yet).
\item Add several test items, both grammatical and ungrammatical,
to the file ``test.items'' which will allow you to check the
correctness of your additions to the grammar.   
\item Run the batch parsing utility again on the file ``test.items'', and
examine the results.
\item Celebrate as appropriate.
 \end{enumerate}\end{enumerate}
\end{document}
