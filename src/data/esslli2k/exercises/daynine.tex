\documentclass[10pt]{article}
\setlength{\evensidemargin}{0in}
\setlength{\oddsidemargin}{0in}
\setlength{\textwidth}{6.5in}
\setlength{\topmargin}{-0.5in}
\setlength{\textheight}{9in}
\pagestyle{empty}
\parindent=0pt

\begin{document}
\begin{center}
\textbf{Advanced Grammar Engineering using HPSG: DAY 4}\\
ESSLLI 2000
\end{center}

\medskip
\textbf{Goals}
\begin{enumerate}
  \item Extend the English grammar to cover some long-distance dependencies.
  \item Explore a binary-branching variant of the syntactic rules.
\end{enumerate}

\medskip
\textbf{Background}

Like many languages, English has constructions where at least one of the
syntactic arguments of a predicate does not show up right beside its
predicate, but somewhere farther away.  One class of these constructions
is called {\it long-distance dependencies}, illustrated by the following
examples:
\begin{enumerate}
  \item {\it That cat, the dog wanted me to give to the aardvark.}
  \item {\it Examples like this, I'm sure nobody ever really says.}
\end{enumerate}
These long-distance dependencies are discussed in more detail in Chapter
15 of the Sag and Wasow textbook; for today's exercise, it will be enough
to remember that a sentence-initial phrase like {\it that cat} can be
analyzed as providing the {\it filler} for a {\it gap} (a missing argument)
somewhere later in the sentence.

\medskip
\textbf{Exercises}
\begin{enumerate}
\item If you wish to continue with your English grammar from Tuesday, feel
free to do so.  But you can also begin today's exercise by checking out a
new version of the grammar which includes semantics:\\

        \verb|  cvs checkout grammar7|\\

Prepare the basic machinery for handling long-distance dependencies.
Our approach will be to remove a complement from the COMPS list
of a predicate using a new syntactic rule, store it for awhile in a GAP 
attribute, and then retrieve it at the top of the phrase structure tree 
using a new syntactic rule.  

\item First, in the file ``types.tdl'' add the feature GAP to the type {\it
syn-struc}, and make its value be of type {\it *dlist*}, for reasons that will
become clear in a moment.  Then consider the two types of {\it syn-struc}; we
will assume that all lexical items have an empty GAP list.  Make this so.

\item Phrases may contain a gap, and that gap might come from any of the
daughters of a phrase, so we use the same difference-list mechanism to collect
up these values that we used for ORTH and RELS earlier in the week.  Add a
subtype of the type {\it unary-rule} called {\it unary-rule-passgap} which
makes the phrase's GAP be unified with the GAP of its only daughter, and add a
subtype of the type {\it binary-rule} called {\it binary-rule-passgap} which
appends the GAP values of its two daughters.  Finally, directly modify the
type {\it ternary-rule} to append the GAP value of its three daughters.
Modify the types {\it unary-head-initial} and {\it binary-head-initial} to
inherit from these new types, and add a subtype of {\it binary-head-final}
which does too.

\item Still in the file ``types.tdl'', add two new rule types {\it
unary-head-initial-startgap} and {\it binary-head-initial-startgap} which have
a non-empty GAP whose single element is a {\it syn-struc}.  Since this
is where we introduce the missing argument, and we assume that only one gap
per sentence is needed, all of the ARGS values should have an empty
GAP value.  (Recall that you can use the notation ``$<$! !$>$'' for an empty
difference-list.)

\item Now move to the file ``rules.tdl'' and modify the rule {\it
head-specifier-rule} to inherit from the type {\it binary-head-final-passgap}
- the rest of the existing rules can stay the same, since we've built the
gap-passing machinery into their types.  

\item Still in ``rules.tdl'', add a new unary rule which inherits from the
type {\it unary-head-initial-startgap}, and which identifies the single
complement of its daughter with the single element in the GAP list of the 
mother.

\item Finally, add a new rule called {\it head-filler-rule} of type {\it
binary-head-final} whose first daughter is identified with the single value of
the GAP attribute on the second daughter.  This is the rule that will combine,
for example, {\it that cat} with {\it the dog chased} to build {\it That cat
the dog chased.} Save your files, then reload the grammar, and check that you
can parse the example {\it That cat the dog chased.}  If not, make the
required repairs, until you can parse this example.

\item Now consider the example {\it That dog the cat gave to the aardvark.}
You will have to add another rule to ``rules.tdl'', this one a binary rule
that inherits from the other gap-starting type that we created above.  And
you may need another binary rule to handle {\it To that cat the dog gave
the aardvark.}  Once these are parsing, try testing your analysis on the
batch file ``gap.items''.  You may want to add an additional constraint
to the {\it head-modifier-rule} which prevents modifiers from having a
non-empty GAP value.  You will also discover that you need to add a
constraint to the definition of {\it start-symbol} in the file ``roots.tdl''.

\item {\bf [Advanced]} Introducing three new grammar rules that do almost the
same thing (putting an argument into GAP) is rather unsatisfying.  Try
reworking the syntactic rules so that complements are always picked up one at
a time, and then see if you can cover all of the data in ``gap.items'' using
only one gap-starting rule.

\end{enumerate}
\end{document}






