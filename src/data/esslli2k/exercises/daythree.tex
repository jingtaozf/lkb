\documentclass[10pt]{article}
\setlength{\evensidemargin}{0in}
\setlength{\oddsidemargin}{0in}
\setlength{\textwidth}{6.5in}
\setlength{\topmargin}{-0.5in}
\setlength{\textheight}{9in}
\pagestyle{empty}
\parindent=0pt

\begin{document}
\begin{center}
\textbf{An Introduction to Grammar Engineering using HPSG: DAY 3}\\
ESSLLI 2000
\end{center}

\medskip
\textbf{Goals:}
\begin{enumerate}
\item Learn more about typed feature structures and unification.
\item Finish treatment of modification.
\item Improve analysis of agreement
\item Capture generalizations in lexicon and rules
\end{enumerate}

\smallskip
\textbf{Exercises:}
\begin{enumerate}
\item Extend the grammar to provide an analysis of modification, admitting
sentences like \emph{The dog barks near the cat}.  We introduce a new head
feature \textsc{mod} and a new syntactic rule for modifiers, and we make use
of the notion of underspecification.

\item The constraints we added yesterday to enforce subject-verb and
determiner-noun agreement get the facts pretty much right, but from a grammar
engineering point of view leave a lot to be desired. 

\begin{itemize}
\item Since the feature \textsc{agr} is introduced on the type \emph{pos}, all 
kinds of words will have an agreement feature.  However, in English only
determiners and nouns have agreement information.  Unused features
unnecessarily increase the size of the grammar and can make errors more
difficult to track down. Modify your grammar so that the feature \textsc{agr}
only appears on \emph{det} and \emph{noun}, and not on \emph{verb} or
\emph{prep}.  To do this, you will need to add a new type,
\emph{agr-pos}, which is a subtype of \emph{pos} and a supertype of \emph{det} 
and \emph{noun}. Be sure to check your work using the batch parse mechanism.

\item The intuition behind determiner-noun agreement is that the \textsc{agr}
value of the noun must be the same as that of its specifier.  In our grammar,
though, the \textsc{agr} value of the noun and the \textsc{agr} value of the
specifier are stipulated separately.  Use re-entrancies to eliminate this
redundancy and capture the generalization underlying agreement. Again, verify
your changes by parsing your test sentences.

In case the syntax of re-entrancies in TDL is still confusing, here's an
example:
\begin{verbatim}
x := y &
  [ F #1 & z,
    G #1 ].
\end{verbatim}
This definition means that the value of the feautre \textsc{f} is \emph{z},
and the value of the feature \textsc{g} is the same as that of the feature
\textsc{f}. 
\end{itemize}

\item Those of you who are familiar with HPSG may be wondering why there is no
Head Feature Principle in the grammar. Even if you aren't, you may have
noticed that in each rule the \textsc{head} value of the whole phrase is
always the same as the \textsc{head} value of one of the \textsc{arg}s. The
argument which contributes the \textsc{head} value to the whole phrase is
known as the \emph{head} of the phrase. For some kinds of phrases, the head
daughter is the first daughter, and for some it's the last daughter. Rearrange
the hierarchy of rules to capture this distinction between head initial
phrases and head final phrases.
\begin{itemize} 
\item In \texttt{types.tdl}, add three new types:
\begin{verbatim}
head-initial := phrase &
 [ HEAD #head,
   ARGS < [ HEAD #head ] ...> ].
head-final := phrase &
 [ HEAD #head,
   ARGS < syn-struc, [ HEAD #head ] > ].
root-hf := root & head-final.
\end{verbatim}
(We will have more to say about the types \emph{root} and \emph{root-hf} later
in the course.)
\item Modify the rules in \texttt{rules.tdl} to inherit from these new types.
For example, the \emph{head-complement-rule-0} should look like:
\begin{verbatim}
head-complement-rule-0 := head-initial & 
[ SPR #a,
  COMPS < >,   
  ARGS < word & 
         [ SPR #a, 
           COMPS < > ] >  ].
\end{verbatim}
We call this rule head initial, even though it has only one daughter, since
the other head complement rules are also head initial.  Head final rules, like
the \emph{head-specifier-rule}, should inherit from \emph{head-final}.  The
feature \textsc{head} should not have to be mentioned in \texttt{rules.tdl} at
all.
\end{itemize}
\item Using the same strategy, find and eliminate more redundant
specifications in the grammar.  Improve the organization of the type hierarchy
to make it easier to add new words that are similar to words already in the
lexicon. As a place to start, take a look at the lexical entries for nouns,
and note that the same information is stated again and again in each entry.
State those generalizations as constraints on a new type, \emph{noun-word},
which every noun lexical entry inherits from. As you work, use the batch parse
facility to make sure none of your modifications have damaged the coverage of
the grammar.
\end{enumerate}
\end{document}
