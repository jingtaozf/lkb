\documentclass[10pt]{article}
\setlength{\evensidemargin}{0in}
\setlength{\oddsidemargin}{0in}
\setlength{\textwidth}{6.5in}
\setlength{\topmargin}{-0.5in}
\setlength{\textheight}{9in}
\pagestyle{empty}
\parindent=0pt

\begin{document}
\begin{center}
\textbf{An Introduction to Grammar Engineering using HPSG: DAY 4}\\
ESSLLI 2000
\end{center}

\medskip
\textbf{Goals:}
\begin{enumerate}
\item Examine the nature of rules and their interaction with processing.
\item Complete the generalization of the lexicon using types.
\item Introduce lexical rules into the grammar.
\end{enumerate}

\smallskip
\textbf{Exercises:}
\begin{enumerate}
\item If you have already added modification to your grammar so that the
desired results are produced for the batch test file ``mods.item'', then
you can keep your current grammar for today's exercises.  But if you have
not yet completed your work on modification, please check out an updated
version for today's exercises by typing the following command in the ``xterm''
window:
\begin{verbatim}
		cvs checkout grammar4
\end{verbatim}

\item Types provide us with a useful tool for expressing generalizations
and eliminating redundancy.  Yesterday we introduced additional types to 
capture the Head Feature Principle in the definitions of the grammar rules,
and today we will carry this idea into the lexicon.  You may have already 
started this work yesterday, but today you should try to eliminate all 
redundancy from the lexical entries in the file ``lexicon.tdl''.  
\begin{enumerate}
\item Introduce subtypes of the type {\it word} for nouns, determiners, 
and verbs, adding any constraints which are true for all instances of each
type:
\begin{verbatim}
		noun-word := word & [ ... ].
		det-word := word & [ ... ].
		verb-word := word & [ ... ].
		prep-word := word & [ ... ].
\end{verbatim}

\item Introduce subtypes of the noun-word type for singular and plural nouns,
and do the same for determiners.
\item Introduce subtypes of the verb-word type whose instances select for
third-singular or non-third-singular subjects for present-tense verbs, and 
an additional subtype of the verb-word type for past-tense verbs.
\item Introduce subtypes of the verb-word type to distinguish intransitives, 
transitives, and the two types of ditransitive verbs.
\item Take advantage of the notion of multiple inheritance to introduce 
lexical types in the file ``types.tdl'' for the verbs in the 
file ``lexicon.tdl'', making use of types from each of these two sets of 
subtypes of the verb-word type.  Remember that the syntax for defining 
multiple inheritance here is as follows:
\begin{verbatim}
		X := Y & Z & [A b].
\end{verbatim}
This definition says that the type X is a subtype of both Y and Z, and X
introduces the feature A with value b.  Note that lexical entries in the
file ``lexicon.tdl'' can only inherit from a single type, so any multiple
inheritance that you introduce must be defined in the file ``types.tdl''.

\item Modify your entries in the file ``lexicon.tdl'' to make use of these
new types. When you are finished with this exercise, each of the definitions in
your ``lexicon.tdl'' file should consist of the name of the lexical entry, 
the name of its lexical type, and the orthography.  Everything else should
be defined in the ``types.tdl'' file.  Here is a sample ideal entry:
\begin{verbatim}
dog := noun-word-3sing &
[ ORTH "dog" ].
\end{verbatim}

\end{enumerate}

\item The type hierarchy allows us to capture a number of generalizations
in the grammar, but we need another mechanism in order to represent the
similarities among morphologically related words which have distinct 
syntactic properties.  We introduce the notion of a lexical rule, which 
derives a new word from an existing word.  For today's exercise, we will 
capture the relationship between the two varieties of the verb {\it gave} 
which appear in the following two examples: 

{\it That dog gave the cat to those dogs.\\
That dog gave those dogs the cat.}

\begin{enumerate}
\item If you are continuing with your grammar from yesterday, then do
the following, to pick up some additional machinery needed for the exercise:
\begin{verbatim}
		cd grammar2
		cvs update
\end{verbatim}
If you are using grammar3, then you already have this additional machinery.

\item Move the {\sc args} attribute from {\it phrase} to a higher type which 
includes both words and phrases.

\item Open the empty file ``lrules.tdl'' and add a new type which has the
following structure:
\begin{verbatim}
dative-shift-lrule := word &
[ HEAD ...,
  SPR ...,
  COMPS ...,
  ARGS < word & [ ... ] > ].
\end{verbatim}
\item Fill in any necessary constraints for each attribute, so that the rule 
takes as its single argument a ditransitive verb with two NP complements, and
produces a ditransitive verb with an NP complement and a PP complement.  You
should replace the type {\it word} in the {\sc args} attribute of this rule 
with the name of the type you chose for the ditransitive verb with two NP 
complements.
\item Remove your hand-built lexical entry for the NP-PP version of {\it gave}
from the file ``lexicon.tdl'' since you now have a fine lexical rule which
produces this entry for you.

\end{enumerate}

\item {\bf Optional exercise:} Extend the treatment of modification to
admit examples like {\it The large dog barks.}  You will have to add a new
{\it pos} type for adjectives, a lexical entry for {\it large}, and a new
grammar rule which allows modifiers to precede their heads.  Try to avoid
redundancy as much as possible.
\end{enumerate}

\end{document}
