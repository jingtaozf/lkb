\documentclass[10pt]{article}
\usepackage{a4wide}

\pagestyle{empty}
\parindent=0pt

\begin{document}
\begin{tabular}[t]{@{}l@{}}
\textbf{Practical HPSG Grammar Engineering}\\
ESSLLI 1998
\end{tabular}
\hfill
\mbox{\fbox{\textbf{\textsf{\Huge Day One}}}}

\medskip
\textbf{Goals:}
\begin{itemize}
\item Become familiar with the LKB grammar development platform.
\item Learn to extend the English grammar by adding lexical types and entries.
\end{itemize}

\smallskip
\textbf{Exercises:}
\begin{itemize}
\item[A.] Install your own copy of Version 1 of the toy grammar, and bring up
the LKB:
\begin{itemize}
\item[1.] Login using the server {\it top} instead of the usual ESSLLI machine
{\it head}.
\item[2.] In the xterm window, at the Unix prompt, right after logging in:
\par\texttt{cvs checkout toy <Return>}
\item[3.] In the Emacs window:
\par\texttt{<Esc> x lkb}
\item[4.] Load the grammar by clicking the \textsf{Load grammar} button in the 
``Lkb Top'' window, then double-clicking on the directory ``toy'' and on the
file ``script''.  Reassuring messages will appear in the ``Lkb Top'' window,
and a window will pop up showing you the type hierarchy for this small
grammar.
\end{itemize}
\item[B.] Test the system by parsing the sentence \emph{Kim sleeps}:
\begin{itemize}
\item[1.] With the mouse in the ``Lkb Top'' window, click on the button
\textsf{Parse}. 
\item[2.] Click on the menu item \textsf{Parse input\ldots}.
\item[3.] With the mouse in the pop-up ``Sentence'' window, click on the
button \textsf{OK}.
\end{itemize}
The system will parse the sentence and pop up a window containing the parse
tree for the single analysis of this sentence.
\item[C.] Add a lexical entry for your own name:
\begin{itemize}
\item[1.] In the Emacs window, open the file ``lexicon.tdl'' for editing:
(note that ``\texttt{<Con\-trol-x>}'' here means you hold down the
\texttt{<Control>} key while you press the \texttt{x} key)
\par\texttt{<Control-x> <Control-f> toy/lexicon.tdl}
\item[2.] In the lexicon.tdl buffer that you get, copy the three lines that
define the lexical entry for \textbf{kim\_1} and modify your copy to make the
value of \textsc{orth} and the value of \textsc{name} match your own name.
(If you have not used the Emacs editor before, ask an instructor or one of
your neighbors for assistance.)
\item[3.] Save the changed version of the file:
\par\texttt{<Control-x> <Control-s>}
\end{itemize}
\item[D.] Reload the grammar and test the effect of your addition:
\begin{itemize}
\item[1.] In the ``Lkb Top'' window, click on the button \textsf{Load
grammar\ldots} and reload the ``script'' file.
\item[2.] Parse the sentence \emph{Kim likes
$\langle\mbox{your-name}\rangle$,} by clicking on the \textsf{Parse}
button as before, placing the mouse in the ``Sentence'' window,
erasing the existing sentence, and entering the new sentence, then
clicking on \textsf{OK}.
\end{itemize}
\item[E.] Add the transitive verb \emph{admires} to the lexicon:
\begin{itemize}
\item[1.] Add the type of the semantic relation for the verb:
\begin{itemize}
\item Open the file ``types.tdl'', and go to the end of the file, where you
will see the definition of the relation \textsf{r\_sleep} used in the definition of
the verb \emph{sleeps.} 
\item Add a similar definition for \textsf{r\_admire}.
\end{itemize}
\item[2.] Add the lexical entry for \emph{admires} in the file ``lexicon.tdl''
similar to the entry for \emph{likes.}
\end{itemize}
\item[F.]
 Reload the grammar and test the effect of your additions:
\begin{itemize}
\item[1.] First make sure you can parse the sentence \emph{Kim admires Sandy.}
\item[2.] Then add some additional test sentences, both grammatical and 
ungrammatical, to the file ``test.items'', which test the correctness of your
new verb entry.
\item[3.] Call the sentence batch processing utility:
\begin{itemize}
\item In the ``Lkb Top'' window, click on the button \textsf{Parse} and
then on the menu item \textsf{Batch parse\ldots}, which will pop up a window
asking you for the file to be processed.
\item Click on the file ``test.items'' in your grammar directory. then hit
the button \textsf{OK}.  This will pop up a new window asking you for the name
of the file where the results of the batch run will be stored.
\item Enter the name ``test.results'' and hit the \textsf{OK} button.
The system will print the message \texttt{Parsing test file} in the ``Lkb
Top'' window when it starts, and will print the message \texttt{Finished test
file} when it is done.
\item Open the file ``test.results'' and inspect the parsing results.
\end{itemize}
\end{itemize}
\item[G.] Add the types and lexical entries needed for determiner-noun phrases:
\begin{itemize}
\item[1.] In the file ``types.tdl'', look at the type definitions for \textsf{pn-lxm} and \textsf{tv-lxm}, then add two new types: 
\begin{itemize}
\item one called \textsf{cn-lxm} for common nouns, which have a
 non-empty \textsc{spr} value whose \textsc{head} is of type
\textsf{det}.
\item one called \textsf{det-lxm} for determiners, with \textsc{head} value \textsf{det} (where the grammar already helpfully has the type \textsf{det}
defined as a subtype of \textsc{pos}).
\end{itemize}
\item[2.] In the same file, at the end of the file, add two new subtypes of
\textsf{reln} for the semantic relations needed in the lexical entries for 
\emph{the} and \emph{book}:
\begin{itemize}
\item one called \textsf{r\_book} 
\item one called \textsf{r\_the}
\end{itemize}
\item[3.] In the file ``lexicon.tdl'' add two new lexical entries:
\begin{itemize}
\item one called \textsf{the\_1} of type \textsf{det-lxm}
\item one called \textsf{book\_1} of type \textsf{cn-lxm} 
\end{itemize}
\item[4.] Reload the grammar, and test your additions by parsing the sentence
\emph{Kim likes the book}  If it parses correctly then go on to the
next step.  Otherwise, correct your revisions, and test again. 
\item[5.] Add several test items, both grammatical and ungrammatical,
to the file ``test.items'' which will allow you to check the
correctness of your additions to the grammar.   
\item[6.] Run the batch parsing utility again on the file ``test.items'', and
examine the results.
\item[7.] Send a copy of your results to the instructors, by executing the
following command at the Unix prompt:
\par\texttt{submit toy/test.results}
\item[8.] Celebrate as appropriate.
 \end{itemize}\end{itemize}
\end{document}
