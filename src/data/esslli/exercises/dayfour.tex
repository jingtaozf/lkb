\documentclass[10pt]{article}
\usepackage{a4}

\pagestyle{empty}
\parindent=0pt

\begin{document}
\begin{tabular}[t]{@{}l@{}}
\textbf{Practical HPSG Grammar Engineering}\\
ESSLLI 1998
\end{tabular}
\hfill
\mbox{\fbox{\textbf{\textsf{\Huge Day Four}}}}

\bigskip
\bigskip
\textbf{Goals:}
\begin{itemize}
\item Learn to use the parse chart and parse tree tools.
\item Extend the grammar to include sentences with passive verbs.
\item Develop a treatment of optionality in complementation.
\end{itemize}

\smallskip
\textbf{Exercises:}
\begin{itemize}
\item[A.] Get your own copy of today's grammar, which is called ``pacifier'', and load it.
\item[B.] Modify the grammar to provide an appropriate label for
adjectives in parse trees.
\begin{itemize}
\item[1.] Parse the sentence \emph{Kim likes the red book}, and examine the parse trees produced.  Note that the parse node dominating the adjective \emph{red} is labeled with a question mark.
\item[2.] Modify the file ``parse-nodes.tdl'' to add a label \textsf{ADJ} for adjectives, analogous to that for determiners.
\end{itemize}
\item[C.] Expand the lexicon and grammar to provide analyses for sentences with passive verbs.
\begin{itemize}
\item[1.] Examine the file ``lrules.tdl'' and study the lexical rule
\textsf{dative-shift} which relates the ditransitive verb lexeme \emph{give} as in \emph{Kim gives Sandy a book} to the NP-PP variant of \emph{give} in \emph{Kim gives a book to Sandy}.
\item[2.] Write a similar rule that takes the past-participle form of a
transitive verb and produces a passive entry,  For example, the input
to the passive rule would be a word like \emph{admired} as in \emph{Kim has
admired Sandy}, and the output would be the word \emph{admired} used in
\emph{Sandy was admired by Kim}.
\item[3.] Add some test examples (grammatical and ungrammatical) to the file ``test.items'' and do a batch parse to check the accuracy of your implementation.
Include examples with modifiers and passives together.
\end{itemize}
\item[D.] Modify your analysis of passives to make the \emph{by}-phrase optional.
\begin{itemize}
\item[1.]  Develop an analysis which will allow the grammar to parse sentences like \emph{Kim was admired}.  As you found yesterday, it is worth considering whether to use lexical rules or syntactic rules (or both) for this problem.
\item[2.] Make the necessary improvements, and test your grammar on a
variety of examples.
\item[3.] Add some test examples to the file ``test.items'' and do a batch parse to test the system.
\item[4.] Send a copy of your results to the instructors, by executing the
following command at the Unix prompt:
\bigskip
\par\texttt{submit pacifier/test.results}
\bigskip
\end{itemize}
\item[E.] (Optional) Extend the grammar to include passive verb phrases as post-nominal modifiers.
\begin{itemize}
\item[1.] Consider the example \emph{The book admired by Sandy is red}.  Note that the passive verb phrase \emph{admired by Sandy} modifies \emph{book} in the same way that prepositional phrases do.
\item[2.] Make the desired improvements in your grammar, then test and
revise as needed.
\item[3.] Add some test examples to the file ``test.items'' and do a batch parse to test the system.  
\item[4.] Send a copy of your results to the instructors, by executing the
following command at the Unix prompt:
\par\texttt{submit pacifier/test.results}
\end{itemize}
\end{itemize}
\end{document}
