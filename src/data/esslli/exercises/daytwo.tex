\documentclass[10pt]{article}
\usepackage{a4}

\pagestyle{empty}
\parindent=0pt

\begin{document}
\begin{tabular}[t]{@{}l@{}}
\textbf{Practical HPSG Grammar Engineering}\\
ESSLLI 1998
\end{tabular}
\hfill
\mbox{\fbox{\textbf{\textsf{\Huge Day Two}}}}

\medskip
\textbf{Goals:}
\begin{itemize}
\item Gain a better understanding of the interactions among constraints.
\item Extend the coverage of the toy grammar by adding grammar rules.
\item Learn to use the grammar debugging tools in the LKB.
\end{itemize}

\smallskip
\textbf{Exercises:}
\begin{itemize}
\item[A.] Get your own copy of today's grammar, which is called
``rattle'', and load it.
\item[B.] Add the missing inflectional lexical rule to allow plural common nouns. 
\begin{itemize}
\item[1.] Open the file ``rattle/infl-rules.tdl'' and add the rule \textbf{pl-n-irule}, which will be similar to the singular noun rule and the rule for third-singular present tense verbs.
\item[2.] Test the system by parsing the sentence \emph{Kim admired the books}.
\end{itemize}
\item[C.] Expand the lexicon and grammar to parse sentences with ditransitive verbs.
\begin{itemize}
\item[1.] Add the lexical type, lexical entry and semantic type for \emph{give}, which takes two complements.  Note that you only need to add the lexeme (the stem form) of the verb, since inflection is now done on the fly.
\item[2.] Add a grammar rule in the file ``grules.tdl'' which is like the \textsf{hd-comp-rule}, but with two complements instead of one. You will need to examine the ``types.tdl'' file to determine the appropriate type for this new rule to inherit from.
\item[3.] Test the system by parsing the sentence \emph{Kim gives Sandy the book}.
\item[4.] Add some test examples (grammatical and ungrammatical) to the
file ``test.items''  and do a batch parse to check the accuracy of your implementation.
\end{itemize}
\item[D.] Extend the lexicon and grammar by adding prepositional phrase modifiers.
\begin{itemize}
\item[1.] In the file ``types.tdl'' add a lexical type for prepositions.  Be sure to consider whether the type you add is a \textbf{lexeme} which must undergo inflection, or a \textbf{word} which does not inflect.  Use the \textbf{prep} subtype of the part-of-speech (\textbf{pos}) type, and be sure to supply a constraint on the \textsc{mod} attribute for prepositions. 
\item[2.] In the file ``lexicon.tdl'' add a lexical entry for the preposition \emph{in}.
\item[3.] In the file ``grules.tdl'' add a grammar rule \textbf{hd-mod-rule} with the modifier as the non-head daughter.
\item[4.] Test the system by parsing the sentence \emph{Kim sleeps in the park}.
\item[5.] If the sentence parses successfully, then go on to the next step; otherwise, continue to revise the grammar until it can parse this example.
\item[6.] Add some test examples to the file ``test.items'' and do a batch parse to test the system.
\item[7.] Send a copy of your results to the instructors, by executing the
following command at the Unix prompt:
\par\texttt{submit rattle/test.results}
\end{itemize}

\end{itemize}
\end{document}
