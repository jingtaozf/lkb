\documentclass[10pt]{article}
\usepackage{a4}

\pagestyle{empty}
\parindent=0pt

\begin{document}
\begin{tabular}[t]{@{}l@{}}
\textbf{Practical HPSG Grammar Engineering}\\
ESSLLI 1998
\end{tabular}
\hfill
\mbox{\fbox{\textbf{\textsf{\Huge Day Three}}}}

\medskip
\textbf{Goals:}
\begin{itemize}
\item Practice using the grammar debugging tools in the LKB.
\item Improve the precision of the grammar through type constraints.
\item Extend grammar coverage to include more complex noun phrases. 
\end{itemize}

\smallskip
\textbf{Exercises:}
\begin{itemize}
\item[A.] Get your own copy of today's grammar, which is called
``buggy'', and load it.
\item[B.] Correct the grammar to eliminate the unwanted additional parses for sentences with multiple prepositional phrases modifying verb phrases.
\begin{itemize}
\item[1.] Parse the sentence \emph{Kim sleeps in the buggy in the park}, and examine the parse trees produced.  Determine which of the trees should not be produced.
\item[2.] Make the necessary correction(s) to the grammar, so that only one parse is produced for the example.  Use the parse chart and related analysis tools in solving the problem.
\item[3.] Add some test examples to the file ``test.items'' and do a batch parse to check the accuracy of your revisions.
\end{itemize}
\item[C.] Expand the lexicon and grammar to allow PPs to modify phrases with nominal heads in addition to verbal heads.
\begin{itemize}
\item[1.] Examine the type \textsf{prep-wd} to see what constraint(s) must be changed in order to expand the range of modification for PPs to a larger class of substantives.
\item[2.] Make the desired modifications, and then test the result on the sentence \emph{The buggy in the park stopped}, and when this is okay, on the sentence \emph{Kim admired the buggy in the park} which should get two parses.
\item[3.] Add some test examples (grammatical and ungrammatical) to the file ``test.items'' and do a batch parse to check the accuracy of your implementation.
\end{itemize}
\item[D.] Extend the lexicon and grammar by adding bare plural noun phrases.
\begin{itemize}
\item[1.] Consider the example \emph{Kim likes books} and figure out the changes needed in the grammar to provide a parse for this example.
\item[2.] Make the desired improvements, and then test your grammar on
several variations, including bare plurals in subject position, with
dative verbs, and as the object in a prepositional phrase.
\item[3.] Now test your analysis on the example \emph{Kim likes buggies in the park} which should have two analyses.  If your grammar is correct, then do the next step; otherwise, make the necessary repairs.
\item[4.] Add some test examples to the file ``test.items'' and do a batch parse to test the system.
\item[5.] Send a copy of your results to the instructors, by executing the
following command at the Unix prompt:
\par\texttt{submit buggy/test.results}
\end{itemize}
\item[E.] (Optional) Extend the grammar to include prenominal adjective modifiers.
\begin{itemize}
\item[1.] Consider the example \emph{Kim likes the red book} and figure out the changes needed in the grammar to provide a parse for this example.
\item[2.] Make the desired improvements, then test and revise your grammar.
\item[3.] Once the above example parses, try the example \emph{Kim likes red books in the park} and check to see that you get the desired number of parses.
\item[4.] Add some test examples to the file ``test.items'' and do a batch parse to test the system.  Be sure to include order variation for all elements of the noun phrase in the ungrammatical examples.
\item[5.] Send a copy of your results to the instructors, by executing the
following command at the Unix prompt:
\par\texttt{submit buggy/test.results}
\end{itemize}
\end{itemize}
\end{document}
